\documentclass{article}
\usepackage[utf8]{inputenc}
\usepackage[top=3cm, bottom=3cm, left=3cm, right=3cm]{geometry}

\title{BrusselsBook}
\author{Hakim Boulahya & Youcef Bouharaoua}
\date{April 2016}

\begin{document}

\maketitle

\section{Modèle entité-association}

\section{Contraintes d'intégrité}

\begin{itemize}
\item L'\textit{UID} Et L'\textit{EID}  sont respectivement  uniques à chaque \textsl{User} et \textsl{Establishement}  .

\item L'\textit{EmailAdress} est spécifique à chaque \textsl{User}.

\item Le \textsl{User} doit impérativement être connecté pour pouvoir effectuer un des \textsl{Comment}.

\item Le \textit{Score} d'un \textsl{Comment} varie entre zéro et cinq .

\item La taille maximale  du \textit{Text} du \textsl{Comment} est fixé à deux-cents  caractères .

\item La taille du \textit{Password} du \textsl{User} varie entre quatre et seize caractères.

\item Le \textsl{User} a le droit d'écrire un \textsl{Comment} par jour par \textsl{Establishement}.

\item Le \textsl{User} ne peut apposer qu'une seule fois le même \textsl{Tag} sur un même \textsl{Establishement} .

\end{itemize}

\section{Remarques}
Pour la question des demi-jours deux solutions sont envisagées , soit utiliser ce qui a été proposé dans le modèle entité association ,
c'est à dire utiliser un attribut appartenant à l'entité \textsl{Etablishement} et le type de cet attribut est un tableau ,
soit utiliser une entité \textsl{DaysOff} qui sera en relation avec l'\textsl{Etablishement} . 

\section{Modèle relationnel}
\noindent
\paragraph{}
Etablissement(EID, Name, Latitude, Longitude, PhoneNumber)
\paragraph{}
Restaurant(EID, PriceMinimum, PriceMaximum, BanquetPlaces, HasTakeaway, MakeDelivery, \textsl{WebSite})
\paragraph{}
Bar(EID, CanSmoke, MakeRestoration)
\paragraph{}
Hotel(EID, NoStars, NoRooms, PriceForTwo)
\paragraph{}
Address(EID, Street, No, Locality, PostalCode)

\begin{itemize}
    \item Restaurant, Bar.EID, Cafe.EID et Address.EID reference Etablissment.EID.
\end{itemize}
\paragraph{}
User(UID, EmailAddress, Username, Password, RegistrationDate) 
\paragraph{}
Administrator(UID) 

\begin{itemize}
    \item Administrator.UID fait référence a User.UID
\end{itemize}
\paragraph{}
Comment(EID, UID, Date, Score, Text) 
\paragraph{}
Tag(Name, CreationDate)



\end{document}

